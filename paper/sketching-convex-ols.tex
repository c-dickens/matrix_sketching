\documentclass[twoside]{article}
%\usepackage{aistats2018}
 % If your paper is accepted, change the options for the package
% aistats2018 as follows:
%
%\usepackage[accepted]{aistats2018}

%
% This option will print headings for the title of your paper and
% headings for the authors names, plus a copyright note at the end of
% the first column of the first page.

%% Useful packages
\usepackage{a4wide}
\usepackage{amsmath}
\usepackage{amssymb}
\usepackage{amsthm}
\usepackage{mathtools}
\usepackage{graphicx}
\usepackage{subcaption}
\usepackage{multirow}
\usepackage{adjustbox}
\usepackage{color}
\usepackage{url}
\usepackage{censor}
\usepackage[normalem]{ulem}
\graphicspath{{../figures/}}
\newcommand{\prob}{\mathbb{P}}
\newcommand{\R}{\mathbb{R}}
\newcommand{\E}{\mathbb{E}}
\newcommand{\V}{\mathbf{Var}}
\newcommand{\eps}{\varepsilon}
\newcommand{\mat}[1]{\boldsymbol{#1}}
\newcommand{\nnz}[1]{\text{nnz}(#1)}

\theoremstyle{definition}\newtheorem{thm}{Theorem}[section]
\theoremstyle{definition}\newtheorem{mydef}[thm]{Definition}
\theoremstyle{definition}\newtheorem{rem}[thm]{Remark}
\theoremstyle{definition}\newtheorem{prop}[thm]{Proposition}
\theoremstyle{definition}\newtheorem{example}[thm]{Example}
\theoremstyle{definition}\newtheorem{claim}[thm]{Claim}
\theoremstyle{definition}\newtheorem{Qu}[thm]{Question}
\theoremstyle{definition}\newtheorem{Lemma}[thm]{Lemma}
\theoremstyle{definition}\newtheorem{Cor}[thm]{Corollary}
\theoremstyle{definition}\newtheorem{Fact}[]{Fact}

%% Useful commands
\DeclareMathOperator*{\argmin}{argmin}

%\author{Author, A.}
\title{Fast Preconditioning for Constrained Least Squares}
\author{Charlie Dickens}
\date{}


\begin{document}
\maketitle

% If your paper is accepted and the title of your paper is very long,
% the style will print as headings an error message. Use the following
% command to supply a shorter title of your paper so that it can be
% used as headings.
%
%\runningtitle{I use this title instead because the last one was very long}

% If your paper is accepted and the number of authors is large, the
% style will print as headings an error message. Use the following
% command to supply a shorter version of the authors names so that
% they can be used as headings (for example, use only the surnames)
%
%\runningauthor{Surname 1, Surname 2, Surname 3, ...., Surname n}

%\twocolumn[

%\aistatstitle{Sparse Embeddings for Least Squares Regression with Convex Constraints}

%\aistatsauthor{ Author 1 \And Author 2 \And  Author 3 }

%\aistatsaddress{ Institution 1 \And  Institution 2 \And Institution 3 }
%]

\begin{abstract}
  Matrix sketching is a dimensionality reduction technique to make large datasets
  more manageable and has recently been appleid to solving a variety of
  convex constrained regression problems.
  There are two well-known sketching methods which can solve such
  regression problems; one being a sketch-and-
  solve type approach, and the other being an iterative scheme known as \textit{
  Iterative Hessian Sketching}.
  First we show that the
  Count Sketch \cite{clarkson2013low}, fits ideally within the iterative method
  and then give a suite of experiments to compare the performance of
  the CountSketch compared to competing methods.
  Our empirical results demonstrate that the Count Sketch is an extremely efficient
  sketch which offers \textit{computational} improvements over other sketching
  methods in this setup.
  \textit{Add more subject to experimental conclusion.}
\end{abstract}

NB. MAKE THE EMPHASIS THAT WE WANT TO SEE
HOW AGGRESSIVELY ONE CAN DOWNSAMPLE AND
STILL MAINTAIN A GOOD PRECONDITIONER IN
THE IHS SKETCHING MODEL.
PROVIDE THE SKETCH AND SOLVE MODEL FOR
COMPARISON.

Testing from Atom second time.

\section{Introduction} \label{sec: intro}
%!TEX root = sketching-convex-ols.tex

The problem of handling the growing volume of data is of increasing
importance, particularly in terms of designing theoretically attractive
algorithms
which are efficient for the practitioner.
In light of this, matrix sketching has recently emerged as a method for
 approximating certain
linear algebra algorithms for popular data analysis primitives.
The main idea is that properties of a data matrix can be well-understood
through a
smaller summary of the data which is much easier to use in computations and
can offer huge computational savings.
Let $A \in \R^{n \times d}$ be the input sample-by-feature matrix with
associated target vector $b \in \R^n$ and assume that $n \gg d$.
Additionally, let $\mathcal{C}$ denote a set of convex constraints and the
task is to find

\begin{equation} \label{eq: convex-ols-problem}
  x_{OPT} = \argmin_{x \in \mathcal{C}} \frac{1}{2} \|Ax-b\|_2^2.
\end{equation}
\noindent In the large data setup we assume that $n$ is sufficiently large so
that solving Equation (\ref{eq: convex-ols-problem}) exactly is not possible.
As a result, one needs to exploit some notion of approximation to solve this
problem efficiently.
Two competing methods of approximation through matrix sketching are the
 \textit{sketch-and-solve} and \textit{iterative Hessian sketching} models which we
  detail below.



\begin{itemize}
  \item \textit{Sketch-and-solve}:
  The sketch-and-solve approaches output estimates found by solving

  \begin{equation}
    \hat{x} = \argmin_{x \in \mathcal{C}} \| S(Ax - b) \|^2
  \end{equation}
  by sampling a random linear transformation $S \in \R^{m \times n}$ from a
  sufficiently well-behaved distribution of matrices with $m \ll n$.
  Now that $m$ is much smaller than $n$, the $m \times d$ dimensional problem
  is small enough that it can be solved exactly.
  In fact, for appropriately chosen $S$, one can show that the \textit{
  cost} of a regression problem can be aproximated accurately in the sense
  that $\| A  \hat{x} - b \|^2 \le (1 + \eps) \|A x_{OPT} - b\|^2$ where
  $\eps$ is an accuracy parameter and the result holds with high constant
  probability.
  It is assumed that the projection dimension $m$ is chosen sufficiently small
  relative to $n$ that the smaller problem is efficiently solvable and that
  the main computational bottleneck is the time it takes to compute the
  summary $SA$.
  We detail the computational complexity of this approach in Section
  \ref{sec: preliminaries}.

  \item \textit{Iterative Hessian Sketching}:
  This approach exploits the quadratic program formulation of Equation
  (\ref{eq: convex-ols-problem}) and uses the random projections to accelerate
  expensive computations in the problem setup.
  In addition, one can argue that the summary $SA$ is acting as a preconditioner
  for the original problem.
  In contrast to the sketch-and-solve approach, the aim here is to define an
  iterative scheme through which one gradually refines the estimate in order
  to descend to the true solution of the problem outlined in Equation (\ref{eq: IHS}).

  \begin{equation} \label{eq: IHS}
    x^{t+1} = \argmin_{x \in \mathcal{C}}  \frac{1}{2} \|S^{t+1} A
    (x - x^t) \|^2 - \langle A^T (y - Ax^t), x - x^t \rangle
  \end{equation}
  The benefit of this approach is that rather than computing $\|Ax\|_2^2$ one
  instead computes $\| SAx \|_2^2$ which is sufficiently-well concentrated to
  around its mean that the iterative scheme enjoys
  convergence to the optimal solution of the regression problem.
  This is a huge computational saving when $SA$ has $m$ rows (which can be bounded
  independently of $n$) rather than the $n$ of $A$ to compute the quadratic form.

\end{itemize}

\textbf{Related Work.}
Within this family of convex constrained least squares problems are popular
data analysis tools such as ordinary least squares ($\mathcal{C} = \R^d$),
and penalised forms of regression: $\mathcal{C} = \{x : \|x\|_p \le t, p=1,2 \}$
among many others such as Elastic Net and SVM.
For problems such as unconstrained regression or LASSO, the time complexity of
solving the optimisation problem (i.e without the use of cross-validation to
choose hyperparameters) is  $O(nd^2)$.
There are various works which have studied these regression problems from both
the sketch-and-solve and IHS perspectives.



\noindent\textit{Sketch-and-solve}:
Provided that a sketch of a matrix is a \textit{subspace embedding} (defined
in Section \ref{sec: preliminaries}) which roughly requires that norms of vectors
are preserved under the transformation $S$, then the cost
of the regression problem can be approximated up to $(1 \pm \eps)$\textit{
relative error} \cite{woodruff2014sketching}.
Sarl\'{o}s used \textit{Fast Johnson Lindenstrauss Transforms}
to construct subspace embeddings to find
the first $o(nd^2)$ algorithm for solving unconstrained regression
 \cite{sarlos2006improved}.
 In addition, a similar idea was employed by Clarkson and Woodruff
\cite{clarkson2013low} to
generate a subspace embedding which could be found in time proportional to the
number of nonzeros in the matrix (denoted $\nnz{A}$ throughout).
Similar ideas have been extended to ridge regression in
\cite{avron2016sharper} as well as other forms of constrained regression again
found in \cite{woodruff2014sketching}.

A particular strength of the sketch-and-solve approach is that only one pass
of the data and
as such it is attractive for extremely large datasets.
Additionally, the \textit{cost}, or objective value, of a regression can be
approximated up to $(1 \pm \epsilon)$ relative error.
However, a draqback of the sketch-and-solve method is that although one can
accurately preserve the objective value of the original problem,
the projection onto a random subspace yields suboptimal estimators of the
solution vector to the original problem.
Therefore, despite these strengths, it is less well-known how the
 sketch-and-solve
approach fairs when new examples are added which is of considerable
importance to machine learning practitioners.
This has been addressed to some extent in \cite{price2017fast} where the
authors give an $\ell_{\infty}$ guarantee on the distance between the
approximate estimator and the optimal solution vector, yet also shown
was the existence of a pathological
regression instance for the CountSketch as opposed to the SRHT.
As a result, using the CountSketch within constrained regression problems
for solution recovery remains little used and given the potential speedup
from using this sketch, one which is possibly very useful.

A recent work of \cite{dahiya2018empirical} has explored the use of the
CountSketch method for a variety of problems.
Although the authors detail the efficacy of this sketch, they focus on
variants of the ordinary least squares problem, robust regression, and other
linear algebra algorithms in the sketch-and-solve model.
As such, the aim of our work is different as we focus solely on convex
constrained regression problems with a view to comparing the two competing
models as well as potential improvements (or deficiencies) of the CountSketch
compared to other random projections.


% Keep below for a two column paper
% \begin{align*}
%     x^{t+1} = \argmin_{x \in \mathcal{C}}  \frac{1}{2} \|S^{t+1} A
%     (x &- x^t) \|^2 \\
%       &- \langle A^T (y - Ax^t), x - x^t \rangle
% \end{align*}

\noindent\textit{IHS}:
Rather than simply solving $\min \|S(Ax-b)\|^2$ Pilanci and Wainwright
\cite{pilanci2016iterative}
proposed an iterative method based on solving the iterative
scheme defined in Equation (\ref{eq: IHS}).
The benefit of the IHS approach is that after sufficiently many iterates,
the output approximation, $\hat{x}$, has small error from the true
optimal solution.
This idea was later refined to sketch both large ($n$) and high dimensional
problems ($d$) through an iterative primal-dual approach in
 \cite{wang2017sketching} but applied specifically to the case of
  high-dimensional ridge regression:
 however, again, the faster CountSketch transform
was not studied.
In a similar line of work, the IHS method was extended to a wider class
of problems (i.e. those whose objective function is twice-differentiable and
convex) than the constrained least-squares regression
\cite{pilanci2017newton} and hence is beyond the scope of this paper.


\textbf{Contributions.}
Our work seeks to understand the benefits and limitations of the two
above competing sketching methods in convex constrained regression.
First we show theoretically that the CountSketch can be used in the iterative
framework developed by Pilanci and Wainwright.
Secondly, we show that the CountSketch also suffers from the same
 solution-approximation
suboptimality under the (standard) Gaussian design setting which motivates its
use in the iterative setup.
Empirically we demonstrate that the CountSketch performs comparably with
other sketching methods on various synthetic and real datasets:
importantly the required projection dimension behaves better than suggested by
the theory.
Then we show that, in line with the theory, CountSketch is practically a much
faster summary method as it can be computed while `streaming' the dataset.
We then give a series of experiments comparing the types of sketches within the
IHS approach and show that for this class of problems the CountSketch is a good
choice of random projection.
Although using a CountSketch as a preconditioner has been studied before
\cite{dahiya2018empirical}, its use
within the IHS framework is not understood and this work highlights that
significant
speedups can be found by exploiting the CountSketch transform.
Our work is timely as much of the current research on randomized sketching
 algorithms
has focused on the theory and there is still a gap between theoretical and
practical understanding which we aim to narrow.

We organise the paper as follows:
\begin{itemize}
  \item Section \ref{sec: preliminaries} introduce the necessary theory, background,
  and comparison of the various sketching models and methods under consideration.

  \item Section \ref{sec: countsketch-properties} outlines the theory for
  understanding why the countsketch can be used within the iterative sketching
  model.
  All proofs in this section are deferred to Appendix \ref{sec: countsketch-proofs}.

  \item  Section \ref{sec: subspace-embedding-results} provides experiments
  which compare the various sketching methods in terms of how good an embedding
  is returned and the associated computational cost.
  
  \item Sections \ref{sec: countsketch-ihs}, \ref{sec: ihs-lasso} compare the
  use of CountSketch
  within the iterative sketching model, comparing performance to the previously
  used sketches as well as the popular Python machine learning library Sklearn.
\end{itemize}



\section{Preliminaries} \label{sec: preliminaries}
%!TEX root = sketching-convex-ols.tex

Let $A \in \R^{n \times d}$ be the input data matrix and let $m$ be the
desired projection dimension.
The key quantity one is concerned with in understanding random projections of
the kind we are interested in is the notion of a \textit{subspace embedding}
which measures how well the directions of a matrix are captured by a small
space representation of a matrix.
This is formalised in the following definition:

\begin{mydef} \label{def: subspace-embedding}
  A matrix $S \in \R^{m \times n}$ is a \textit{$(1 \pm \eps)$-subspace embedding}
  for the column space of a matrix $A \in \R^{n \times d}$ if for all vectors
  $x \in \R^d, \|SAx\|_2^2 = (1 \pm \eps)\|Ax\|_2^2$.
\end{mydef}

\noindent We define the families of random matrices $S \in \R^{m \times n}$ that
are used to provide subspace embeddings.

\begin{itemize}
  \item \textit{Gaussian} sketch: $G_{ij} \sim N(0,1)$ and the matrix
  $S = G/\sqrt{m}$
  \item \textit{Subsampled Randomized Hadamard Transform (SRHT)}: $S = PHD$
  where $D$ is a diagonal matrix with $D_{ii} \in \{ \pm 1 \}$ each with
  probability $1/2$ independently.
  The matrix $H$ is the recursively defined Hadamard Transform and $P$ is
  a matrix
  which samples rows uniformly at random.
  \item \textit{CountSketch}: initialise $S = \mathbf{0}_{m,n}$ and for every
  column $i$ of $S$ choose a row $h(i)$ uniformly at random.
  Set $S_{h(i),j} = \pm 1$ with probability $1/2$.
  \item{\textit{TO INCLUDE: Sparse Johnson Lindenstrauss Transform (SJLT)}:}
\end{itemize}

\textbf{Time Complexity of sketching methods:}
Each of the above described random projections defines a linear map from $\R^n
\to \R^m$ which naively would take $O(mnd)$.
Despite this, only the Gaussian sketch suffers from the matrix multiplication
time cost because the SRHT exploits the fast Hadamard transform which takes
$O(nd \log n)$ time as it is defined recursively.
Additionally, the CountSketch can be computed by streaming through the matrix
$A$: upon observing an entry $A_{ij}$, the value of a hash bucket defined
by the function $h$ is then updated with either $\pm A_{h(i),j}$ and hence
the time
to compute the CountSketch transform is roughly $O(\nnz{A})$.
In light of this the matrix $SA$ can be viewed both as a linear transform but
also a method of summarising the input data to a small space representation.

\begin{thm}[\cite{woodruff2014sketching}] \label{thm: subspace-embedding-dims}
  Let $A \in \R^{n \times d}$ have full column rank.
  Let $S \in \R^{m \times n}$ be sampled from one of the Gaussian, SRHT, or
  CountSketch distributions.
  Then to achieve the subspace embedding property with probability
  $1 - \delta$ we require:
  $m = O(\eps^{-2}(d + \log(1/\delta)))$ for the Gaussian sketch,
  $m = \Omega(\eps^{-2}(\log d) (\sqrt{d} + \sqrt{n})^2)$ for the SRHT,
  $m = O(d^2/(\delta \eps^2))$ for the CountSketch.
\end{thm}

Theorem \ref{thm: subspace-embedding-dims} highlights an important distinction
to be made between the sketching methods.
Firstly, the projection dimension, $m$, of both the Gaussian and SRHT is dependent
(at worst) $d \text{poly} \log d$ whereas the CountSketch, despite being faster to
apply, depends on $d^2$.
As such, the CountSketch is suboptimal with respect to the projection dimension
as $d^2$ projections are required: \color{red} the reason being that it is the number of buckets required to
ensure that $d$ directions are hashed perfectly into $m$ buckets.
\color{black} In addition, the failure probability of the CountSketch depends \textit{linearly}
upon $\delta$ whereas both the Gaussian and SRHT depend on $\log 1 / \delta$.
This is a significant difference and is sufficient evidence for one to question
the practical efficacy of the CountSketch given that it is theoretically suboptimal
with respect to two important parameters.
Despite the theoretical deficiency of the CountSketch, we give compelling evidence
that it can be used in practise as an extremely fast alternative to both of the
previously mentioned projection methods.

\begin{table}[ht]
\centering
\begin{adjustbox}{}
\begin{tabular}{|c|c|c|c|c|}
  \hline
Sketch Method  & Embedding Dimension ($m$)              &   Projection Time \\
\hline
Gaussian       & $O \left( \frac{d}{\log \frac{1}{\delta} \eps^2}\right)$  &   $O(nd^2)$   \\
SRHT           & $O \left( \frac{d}{\log \frac{1}{\delta} \eps^2} \right) $ &   $O(nd \log n)$ \\
CountSketch    & $O \left( \frac{d^2}{\delta \eps^2} \right)$          &   $O(\nnz{A})$ \\
\hline
\end{tabular}
\end{adjustbox}
\caption{Comparison of sketching method complexity measures}
\label{table: sketch-facts}
\end{table}


\subsection{Setup}
Throughout we consider examples of overconstrained regression in the form of
a data matrix $A \in \R^{n \times d}$ and target vector $b \in \R^{n}$ with
$n \gg d$.
Also given is a set of convex constraints $\mathcal{C}$ and the task is to
find the \textit{optimal estimator}:

\begin{equation}
  x_{OPT} = \argmin_{x \in \mathcal{C}} \frac{1}{2} \|Ax-b\|_2^2.
\end{equation}

We consider two popular models of sketching as well as different types
of random projections.
The two methods are know as \textit{sketch-and-solve} and \textit{iterative
Hessian sketching} (IHS) which we now outline.
For a random projection $S$, the sketch-and-solve approach outputs estimates
of the form:

\begin{equation} \label{eq: sketch-and-solve}
  \hat{x}_S = \argmin_{x \in \mathcal{C}} \frac{1}{2} \|S(Ax-b)\|_2^2.
\end{equation}
\noindent In contrast, the IHS method instead sketches in the following manner:

\begin{equation} \label{eq: ihs}
  x^{t+1} = \argmin_{x \in \mathcal{C}}  \frac{1}{2} \|S^{t+1} A
  (x - x^t) \|^2 - \langle A^T (y - Ax^t), x - x^t \rangle.
\end{equation}

The reason for the differing approaches is due to the following notions of
approximation that we introduce as \textit{cost} and \textit{solution}
approximation.
Let $f(x) = \frac{1}{2}\|Ax-b\|_2^2$ for input parameters $(A,b)$ and let
$x_{OPT}$ denote the optimal solution.
We also use the following notation: $\|x\|_A = \frac{1}{\sqrt{n}}\|Ax\|_2$.

\begin{mydef} \label{def: cost-approx}
  An algorithm which outputs $\hat{x}$ is a $(1 + \eps)$ \textit{cost
   approximation} if $f(\hat{x}) \le (1+\eps) f(x_{OPT})$.
\end{mydef}

\begin{mydef} \label{def: sol-approx}
  An algorithm which returns $\hat{x}$ such that $\|x_{OPT} - \hat{x}\|_A
  \le \eps \|x_{OPT}\|_2$ is referred to as a $\eps$-\textit{solution
  approximation} algorithm.
\end{mydef}

Inspecting Equations (\ref{eq: sketch-and-solve}) and (\ref{eq: ihs}) one
sees that that a key quantity we need to understand is that of
a \textit{subspace embedding} introduced in Definition \ref{def: subspace-embedding}.
Each of the sketches defined in Section \ref{sec: intro} achieves the subspace
embedding criterion which is summarised in Theorem
\ref{thm: subspace-embedding-dims}.
One can immediately see that the subspace embedding obtains a ($1+\eps$)-cost
approximation to the problem

\noindent \textbf{Time Complexity of sketching models.}
The time taken for each of the models is simply the sketch time $T_{\text{sketch}}$
and the time to solve a smaller problem in the `sketch space', $T_{\text{solve}}$.
When the data is compressed into a summary of rougly $d \times d$ then $T_{\text{solve}}$
is $O(d^3)$; note that this can potentially be $O(d^4)$ if we require the full
$d^2$ projections from the CountSketch but this does not seem to occur in
practise as for many datasets $n$ is sufficiently large compared to $d$ that
CountSketch performs comparably with the other sketching methods.
Note that for the IHS model, we must repeat this $O(\log 1/\eps)$ times for
convergence whereas sketch-and-solve is a `one-shot' algorithm.


\textbf{Questions to answer}:
\begin{itemize}
  \item Why solution approimation?
  \item IHS with one iteration.
  \item Order one bound error comment from IHS Eqn 5
  \item Comments on the IHS algorithm, iterations etc.
  \item \sout{Hesian Sketch as preconditioner}
  \item Complexity arguments
  \item need $d + \log n$ in each round for IHS.
  \item How small can the preconditioner be for convergence?
  \item How small can the subspace embedding dimension be for $1\pm \eps$ guarantee
\end{itemize}

\color{red}
It has also been argued in \cite{wang2017sketching} that IHS is acting as
a preconditioned approach to the Newton method.
Indeed, for $f(x) = \frac{1}{2} \|Ax-b\|_2^2$ we see that $\nabla f(x) =
-A^T(b-Ax)$.
In the iterates, the task is to find $u' = \argmin_{x \in \mathcal{C}}
\frac{1}{2}\|SAu\|_2^2 - u^TA^T(b-Ax_t)$.
By solving the unconstrained problem one would obtain $u' = (A^TS^TSA)^{-1}
A^T(b-Ax_t)$ and thus, $u' = - \tilde{H}^{-1} \nabla f(x_t)$.
Then taking $x^{t+1} = \mathcal{P}_{\mathcal{C}} (x^t + u')$ we obtain
$x^{t+1} = \mathcal{P}_{\mathcal{C}} (x^t - \tilde{H}^{-1}\nabla f(x_t))$
which is a projected Newton step with a sketched approximation to
the Hessian.
\color{black}

\subsection{Motivating the IHS: Sketch-and-Solve Suboptimality}

For a variety of the results we asume the standard Gaussian design
stating that $y = A x^* + \omega$ (in contrast results from the sketch-and-solve
model hold with no assumptions on the noise vector $\omega$).
Here, the data $A$ is fixed and there exists an unknown ground truth
vector $x^*$ belonging to some compact $\mathcal{C}_0 \subseteqq \mathcal{C}$.
In addition, the error vector $\omega$ has entries are drawn
by $\omega_i \sim N(0, \sigma^2)$.
In order to compare the output of an algorithm, $\hat{x}$, to that of the
\color{red} `exact' (TERMINOLOGY optimal / best ??!!) \color{black} estimator, $x_{OPT}$ Pilanci and Wainwright
showed that the expected solution error could be bounded above as a function
of the input size and from below as
In addition, it was shown that any estimator $x'$ which observes the pair
$(SA, Sb)$ is provably suboptimal in the sense outlined in Theorem
\ref{thm: clsq-lower-bound}.
The key condition is that a sketching matrix has the following
property (in spectral norm):

\begin{equation} \label{eq: sketch-spectral-property}
  \| \E \left[ S^T (S S^T)^{-1} S \right] \|_2 \le \eta \frac{m}{n}
\end{equation}

\begin{thm}[\cite{pilanci2016iterative}] \label{thm: clsq-lower-bound}
  Let $S \in \R^{m \times n}$ be a sketching matrix which satisfies
  Equation (\ref{eq: sketch-spectral-property}), then any estimator
  based on observing $(SA, Sb)$ has solution error lower bounded as:

  \begin{equation}
    \sup_{x^* \in \mathcal{C}_0} \E_{S,\omega} [\|x^* - x'\|_A^2 ]
     = \Omega \left(
    \frac{\sigma^2}{\eta \min(m,n)} \right).
  \end{equation}
\end{thm}

\begin{rem}
  We note that the constants for Theorem \ref{thm: clsq-lower-bound} are
  explicitly known and are simply the factor $\log(0.5 M_{0.5})/128\eta$
  which we omit for brevity.
  Note that $M_{0.5}$ is the $0.5$-packing number of $\mathcal{C}_0$ in
  the semi-norm $\| \cdot \|_A$.
\end{rem}
Although we omit the details (which can be found in \cite{pilanci2016iterative})
we note that the CountSketch satisfies the spectral condition (\ref{eq: sketch-spectral-property})
and as such, suffers from the same suboptimality as the previously used
sketching methods in the sketch-and-solve model.
This is proved in Theorem \ref{thm: spectral-theorem} and hence motivates the
novel use of CountSketch within the IHS framework rather than the straightforward
sketch-and-solve model.

\begin{prop}[\cite{pilanci2016iterative}] \label{prop: ihs-error-bound}
  For any set $\mathcal{C}$ containing $x^*$, the constrained least-squares
  estimate $x_{OPT}$ has prediction error upper bounded as:

  \begin{equation}
    \E \|x_{OPT} - x^* \|_A^2 \le c \left(\eps_n(\mathcal{\bar{K}}) +
    \frac{\sigma^2}{n}\right).
  \end{equation}
  For the unconstrained case we have the prediction error is $O(\sigma^2 d/n)$.
\end{prop}

The proofs detailing the structural results of why we can use the CountSketch
within the IHS are given in Appendix \ref{sec: countsketch-proofs} however we
outline the key details here.
First we need that the sketches are zero mean and have identity covariance
$\E(S^TS) = I_{n}$ which is shown in Lemma \ref{lem: covariance_matrix}.
Two further properties are required for the error bounds of the IHS.

\begin{mydef}
  The \textit{tangent cone} is the following set:
  \begin{equation}
    K = \{ v \in \R^d : v = tA(x-x_{OPT}), ~ t \ge 0, x \in \mathcal{C}\}
  \end{equation}
\end{mydef}
\noindent We note that the residual error vector for an approximation $\hat{x}$
belongs to this set as $u=A(\hat{x} - x_{OPT})$.
Let $X = K \cap \mathcal{S}^{n-1}$ where $\mathcal{S}^{n-1}$ is the set of $n$-
dimensional vectors which have unit Euclidean norm.
The quanitites we need to understand are:

\begin{align}
  Z_1 &= \inf_{u \in X} \|Su\|_2^2 \\
  Z_2 &= \sup_{u \in X} | u^T S^T S v - u^T v |.
\end{align}
\noindent Here, $v$ denotes a fixed unit-norm $n$-dimensional vector and assume
that $S$ is a subspace embedding for the column space of $A$.
For the IHS conditions to hold it is required that $Z_1 \ge 1-\eps$ and
$Z_2 \le \eps/2$ for $\eps \in (0,1/2)$.
Since $ u \in K \subseteq \text{col}(A)$ and then by the subspace embedding
property we have that $\|Su\|_2^2 = (1 \pm \eps)\|u\|_2^2 = (1 \pm \eps)$ and
hence $Z_1  \ge 1-\eps$ as required.
The second property follows from approximate matrix product with sketching
matrices:
indeed, for a matrix $X$ with only one nonzero $X_{ij}$ then $\|X\|_F = (X_{ij}^2)^{1/2}$
which is exactly $|X_ij|$.
Pad $u$ and $v$ with zeros until they are both $d \times k$ dimensional, to get
$\mat{u}$ and $\mat{v}$ respectively so that we can now apply the matrix product
(Theorem 13 of \cite{woodruff2014sketching} but originally \cite{kane2014sparser}).
This gives $\|\mat{u}^T S^TS \mat{v} - \mat{u}^T\mat{v}\|_F \le 3 \eps \|\mat{u}\|_F \|\mat{v}\|_F$
for $\eps \in (0,1/2)$.
Now, by rescaling $\eps$, recalling that the definitions of $\mat{u},\mat{v},u,v$
mean $\|\mat{u}\|_F = 1, \|\mat{v}\|_F=1$ and then error difference has exactly
one element so therefore reduces to the absolute value of that element, which is
exactly $| u^T S^T S v - u^T v |$ and hence $Z_2 \le \eps$.
Note that although the condition on $Z_2$ looks like a Johnson Lindenstrauss
Transform property, we cannot immediately invoke that for the CountSketch in the
general case due to a lower bound on the column sparsity needed for  a JLT
(discussion of this can be found in \cite{woodruff2014sketching} between Theorems
7 and 8).
Coupled with the fact that the rows of a CountSketch matrix $S$ are sub-Gaussian,
we are now free to use the CountSketch within the IHS framework.

\subsection{Iterative Hessian Sketch}
\begin{enumerate}
  \item{Convergence theorems}
  \item{Sketch dimensions}
  \item{Number of iterations}
\end{enumerate}


\section{CountSketch in IHS: Theory} \label{sec: countsketch-properties}
\subsection{Experimental Outline}

\begin{enumerate}
  \item{\textbf{Summary Time vs Data Density.}  Compare running time for the
  summary computation as a function of data density.
  Do this on synthetic and real datasets.}
  \item{\textbf{Distortion.}  Given a sketch of fixed size, the distortion be
  well-understood.  Again refer to real and synthetic datasets.}
  \item{\textbf{Preliminary IHS experiments.}Do we lose anything by using the
  CountSketch in the IHS scheme? Compare solution error, prediction error}
  \item{\textbf{Case Study - LASSO.} Solve a LASSO
  instance on large datasets, how does the approximation compare to the SKLEARN
  implementation and the sketch-and-solve method}
\end{enumerate}




\subsection{Implementation Details}

All of our experiments are performed in the Anaconda distribution of Python
with only a few extra libraries.
We exploit a fast library for computing the Hadamard
Transform\footnote{https://bitbucket.org/vegarant/fastwht} and the Numba
library\footnote{https://numba.pydata.org/} to accelerate the Python code
so that it runs at comparable speed.
We test our algorithms on real and synthetic data which is summarised in Table
\ref{table: data-facts}.
Our code is made available at
\censor{\url{https://github.com/c-dickens/matrix_sketching}}\footnote{censore for review process}.

\begin{table}[ht]
\centering
\begin{adjustbox}{width=1\textwidth}
%\small

\begin{tabular}{|c|c|c|c|c|c|c|}
  \hline
Dataset                   & Dimensionality       &   Aspect ratio $\frac{d}{n}$ & Density $\nnz{A}/nd$ &  Coherence Ratio & Rank & Source \\
\hline
YearPredictionsMSD       & $(515345,91)$         &   $1.7 \times 10^{-4} $      & 1.0                  &  2608.7                 &  91   &  \cite{Dua:2017}             \\
US Census                & $(5048299, 12)$         & $2.4 \times 10^{-6}$       &
0.5                 &    65.7           &  12 &  \cite{census2000}\\
California Housing       & $(20000, 17)$         &   $8.5 \times 10^{-4} $      &  0.76                 & 4261.4                 & 17    &   \cite{geron2017hands}      \\
Rail2586                 & $(923269, 2586)$      &   $2.7 \times 10^{-3} $      &  0.003                &  Timeout                & 2586    &   \cite{davis2011university} \\
% US Census                & $(5000000,10)$                 &            &            &           &          &      &         &            & \\
\hline
\end{tabular}
\end{adjustbox}
\caption{Comparison of CountSketch and SRHT on real datasets.
Coherence Ratio is the ratio of the largest and smallest leverage scores which
gives a measure of how uniformly distributed the leverage scores are.}
\label{table: data-facts}
\end{table}



\section{Embedding Properties} \label{sec: subspace-embedding-results}
%!TEX root = sketching-convex-ols.tex

As outlined in Section \ref{sec: preliminaries} the key quantity that we would
like to understand is the subspace embedding.
The central questions we seek to answer are: (1) whether the time complexity
in practise matches the theory, (2) how large should the projection
dimension be in order to obtain small error (3) the CountSketch
theoretically requires a larger projection to obtain the subspace embedding
property, is this reflected in practise?

\noindent\textbf{Time Complexity compared to sparsity.}
Only the CountSketch and SRHT are tested because the explicit matrix
multiplication for the Gaussian sketch is exceedingly high which even
for moderately sized inputs will be prohibitive compared to the other two methods.
The CountSketch takes
$O(\nnz{A})$ time to construct a subspace embedding
whereas the SRHT requires $O(nd \log n)$.
Although the SRHT does not scale with the sparsity of the data, for the
practitioner it
will be of interest to know where, if any, there is a threshold at which one of
the transforms is preferable from the perspective of time cost.
To test this claim we generated random matrices of size $n=50000$ with varying $d$
in $\{ 10, 100, 1000,5000 \}$.
This captures a range of aspect ratios seen in the real datasets that we
test as described in Table \ref{table: data-facts}.
The \textit{density} parameter $\rho$ (varied from 0 to 1) denotes the
fraction of nonzeros in the test matrix $A$ which was generated using the
\texttt{scipy.sparse.random(n,d,density=$\rho$)} routine, returning a
matrix of $nd\rho$ nonzero entries chosen independently from a standard normal
distribution.
Both sketching methods are agnostic to the distribution of choice so from a
speed perspective this makes no difference.
A projection dimension of $m = 5d$ was chosen for the summarisation - we note
that  sufficient
sketching dimensions will be discussed in the succeeding section.
As seen in Figure \ref{fig: summary-time-50000}, both transforms scale
consistently with the theory: the CountSketch has increased time to compute the
summary as the density of the data increases and the time taken to compute the
SRHT embedding is generally stable as the density varies.
One key point to take away is that the CountSketch is \textit{an order
of magnitude faster} to compute the embedding compared to the SRHT even as the
density increases;
as $n$ grows to be significantly larger than $d$, this saving could be vast.
Interestingly, our implementations do not show a cross over point for corresponding
settings of the parameters i.e for a fixed $(n,d,m)$ triple the time to compute the
CountSketch summary is always significantly less than that for
the SRHT summary.
This behaviour is consistent across each of the setup values so only those for
the $n = 50000$ are given here.

In addition, the time saving of using the CountSketch over the SRHT is reflected
on the real datasets with the results given in Table
\ref{table: real-data-subspace-embedding}.
CountSketch embeddings are found at a fraction of the time for the SRHT which
allows us to sketch the data in less than 1 second on all but two of the
experimental setups.
Moreover, the (at least factor 10) speedup is in general observed on each of the
real datasets, particularly those which are `tall-and-skinny', which is consistent
with the synthetic data experiments.
For the wider matrices ($d>1000$) the SRHT times out in our
implementation so we report no results.
Although the embedding time difference is quite severe in some cases, we note
that the time complexity of our SRHT implementation is comparable to the test
cases of the Hadamard transform that we use given in its repository.


\begin{figure}
  \centering
\includegraphics[scale=0.75,keepaspectratio]{summary_time_density_50000}
        \caption{$n=50000$ and $d$ from $10,100,1000,5000$.  The legend
        describes the sketch used and the number appended is the value of $d$
        chosen for that particular experiment.
        A projection dimension of $m=5d$ was used.
        The mean time over 5 trials has been reported.}
        \label{fig: summary-time-50000}
\end{figure}





% \begin{figure}
%   \includegraphics[scale=0.25,keepaspectratio]{summary_time_density_10000}
%   \caption{Time to compute summaries for $n=100000$, $d$ given in legend next
%   to sketch name and sketch dimension $m = 5d$ used.} \label{fig: summary-times-vs-sparsity}
% \end{figure}




% \begin{figure}
%     \centering
%     \begin{subfigure}[b]{0.49\textwidth}
%         \includegraphics[scale=0.45,keepaspectratio]{summary_time_density_10000}
%         \caption{$n=100000$}
%         \label{fig: summary-time-100000}
%     \end{subfigure}
%     \begin{subfigure}[b]{0.49\textwidth}
%         \includegraphics[scale=0.425,keepaspectratio]{distortion_vs_cols}
%         \caption{Distortion compared to number of columns}
%         \label{fig: distortion-columns}
%     \end{subfigure}
%     ~ %add desired spacing between images, e. g. ~, \quad, \qquad, \hfill etc.
%       %(or a blank line to force the subfigure onto a new line)
% \end{figure}

% Please add the following required packages to your document preamble:
%


\begin{table}[ht]
\centering
\begin{adjustbox}{width=1\textwidth}
\small

\begin{tabular}{|c|c|c|c|c|c|c|c|c|c|c|c|c|}
  \hline
\multirow{2}{*}{Dataset} & \multicolumn{3}{c|}{CountSketch (time)} & \multicolumn{3}{c}{SRHT (time)} & \multicolumn{3}{c|}{CountSketch (error)} & \multicolumn{3}{c|}{SRHT (error)} \\
                        &     $1d$       &  $2d$            & $5d$          & $1d$          & $2d$        & $5d$         &     $1d$       &  $2d$          &   $5d$         &  $1d$          & $2d$         & $5d$  \\
\hline

Complex & 0.028 & 0.046 & 0.111 & 0.264 & 0.235 & 0.309 & $0.924_5$ & $0.464_5$ & $0.186_5$ & $0.829_5$ & $ 0.367_5$ & $0.0877_5$ \\

Landmark & 0.522 & 1.22 & 3.68 & - & - & - & $0.382_5$ & $0.190_5$ & $0.076_5$ & - & - & - \\

Slice &  0.067  &  0.073  &  0.117 &   1.34   & 1.29 & 1.27 &  $0.004_5$  &  $0.003_5$ &  $0.0006_5$ &   $0.004_5$ & $0.005_5$  &  $0.0005_5$  \\

Rail2586  & 0.897 & 1.31 & 3.15 & - & - & - & \textbf{0.085} & 0.043 & 0.017 & - & - & - \\

California Housing     &   0.001          &  0.001           &   0.002          & 0.018         &  0.022        & 0.017          &   \textbf{0.135}          &  0.024           & 0.008       & \textbf{0.242}          & 0.172         & 0.049 \\


YearPredictionsMSD      &    0.143         &   0.143          &  0.144          &  3.06         &  3.02        &   2.99        &  \textbf{0.031}           &   0.049          &    0.007           &   \textbf{0.033}        &   0.027        &  0.003         \\

US Census   & 0.261 & 0.257 & 0.240 & 7.08 & 7.23 & 6.62 & \textbf{0.032} &  0.152 & 0.037 & \textbf{0.063} & 0.081 & 0.041 \\

Susy  & 0.444 & 0.440 & 0.423 & 10.7 & 10.9 & 10.9 & \textbf{0.200} & 0.103 & 0.028 & \textbf{0.196} & 0.056 & 0.041 \\

\hline

w1a & 0.002 & 0.003 & 0.005 & 0.031 & 0.027 & 0.031 & $0.050_5$ & $0.026$ & $0.010_1$ & 0.047 & $0.026_2$ & $0.009_3$ \\

w2a & 0.001 & 0.002 & 0.004 & 0.032 & 0.033 & 0.037 & \textbf{0.052} & 0.030 & 0.009 & \textbf{0.047} & 0.024 & 0.007 \\

w3a & 0.001 & 0.002 & 0.006 & 0.066 & 0.058 & 0.073 & $0.048_2$ & $0.026_2$ & \textbf{0.001} & 0.057 & $0.024_2$ & \textbf{0.01} \\

w4a & 0.001 & 0.003 & 0.005 & 0.064 & 0.061 & 0.072 & 0.048 & $0.030_1$ & \textbf{0.011} & \textbf{0.049} & 0.024 & 0.001 \\

w5a & 0.002 & 0.003 & 0.005 & 0.126 & 0.133 & 0.138 & \textbf{0.047} & 0.024 & 0.009 & \textbf{0.051} & 0.024 & 0.010 \\

w6a & 0.002 & 0.004 & 0.006 & 0.329 & 0.298 & 0.331 & \textbf{0.053} & 0.022 & 0.010 & \textbf{0.050} &  0.026 & 0.012 \\

w7a & 0.003 & 0.004 & 0.006 & 0.369 & 0.366 & 0.342 & \textbf{0.050} & 0.024 & 0.011 & \textbf{0.049} & 0.023 & 0.010 \\

w8a & 0.005 & 0.006 & 0.010 & 0.956 & 0.930 & 0.994 & \textbf{0.052} & 0.024 & 0.010 & \textbf{0.046} & 0.024 & 0.010 \\
\hline
\end{tabular}
\end{adjustbox}
\caption{Comparison of subspace embedding results using CountSketch and SRHT
on real datasets.
Bold text indicates
the lowest projection dimension $m$ for which rank is preserved at this level
\textit{as well as all projection dimensions larger than $m$} in every trial.
We indicate the number of failures by a subscript: \textit{no} bold text and
\textit{no} subscript indicates that for this $m$ a subspace embedding was
achieved, however, for at least one of the larger projections there was a trial
in which rank was lost (for instance $(\text{w1a, SRHT}, 1d)$).
Observe that the three fail cases are the datasets  with the largest three
aspect ratios.}
\label{table: real-data-subspace-embedding}
\end{table}

% -----------------------------------------------------------------------------


\noindent\textbf{Sufficient Sketching Dimension}
Although our theory shows that a subspace embedding is sufficient to use a random
projection within the IHS scheme, we first need to understand how large the
\textit{sampling factor} $\gamma$ must be relative to $d$ in order to obtain a subspace
embedding.
The constants within the $O(\cdot)$ term must be understood to ensure that $SA$
is of full rank and we investigate how this changes depending on the aspect
ratio of the matrix.
We carry out this experiment on synthetic data and then test again on the real
datasets.
For this set of experiments we fix $n$ and then generate a selection of matrices
of full density
ranging from `tall-and-skinny' to `fat' matrices by choosing $d$ to vary
from roughly $0.05n$ to $0.5n$ over different distributions.
The results from choosing matrices whose entries are standard normal are given
in Figure \ref{fig: distortion-sampling-factor}.
The value of $\eps$ was then measured: this is referred to as the
\textit{distortion} and is the error induced by the approximate matrix product
computation $\|A^T A - A^T S^T S A \|_F/\|A^TA\|_F$.
Experiments were repeated 10 times and the mean distortion has been reported.
In addition, we also measure the rank of the returned matrix $SA$ as it must
be preserved for a true subspace embedding.
The plots indicate that in general, the distortion induced by the CountSketch is
comparable with the Gaussian sketches however both are a constant factor worse
than the SRHT across each of the settings for this distribution.
In addition, the SRHT and Gaussian sketches retain rank in every instantiation
of the experiment yet this is not the case for the CountSketch which retains
rank in all but one (Figures \ref{fig: subspace-125e3-aspectratio} and \ref{fig: subspace-375e1-aspectratio} but at the lowest
sampling factor) of the trials when the aspect ratio is 0.125 or 0.25.
When the aspect ratio exceeds 0.25 we begin to see more rank deficiency for
lower sampling factors and as such require $\gamma > 1.10$ for a CountSketch
subspce embedding but only $\gamma = 1.01$ for SRHT/Gaussian subspace embeddings as
seen in Figure \ref{fig: subspace-375e1-aspectratio}.
This behaviour is exacerbated in the `fat' case, Figure \ref{fig: subspace-5e1-aspectratio},
when $d/n = 0.5$ under which the CountSketch \textit{never} returns a subspace
embedding but both of the other methods are stable in this regime.


The CountSketch achieves comparable distortion to the other methods and constructs
an embedding in vastly less time, however, there is a price to pay for this
computational benefit.
As the aspect ratio increases the number of rank deficient embeddings also
grows; taking $\gamma > 1.05$ is sufficient to ensure that rank is preserved
when $d/n=0.25$ yet the required sampling factor grows considerably with $d/n$.
When $d/n=0.375$ we need roughly $\gamma = 1.125$ for all of the embeddings to
preserve rank for the CountSketch but when $d/n = 0.5$ \textit{none} of the
tested sampling factors up to $\gamma=1.25$ returned a subspace embedding.
This failure of the CountSketch to accurately preserve rank at higher aspect
ratios highlights one deficiency it possesses compared to the other sketches.
Put simply, \textit{CountSketch is more sensitive to higher aspect ratios and
requires a higher sampling factor in this regime compared to the other
sketching methods.}
However, given that the sketches are designed for particularly tall and skinny
matrices, one could argue that CountSketch failing above $d/n = 0.375$ is in
fact not problematic: on all the test instances which fall into the tall-and-
skinny setup, CountSketch retains rank with comparable distortion.

On real datasets we see similar behaviour in that the CountSketch performs well
when the aspect ratio is particularly small but the performance of rank
retention deteriorates as the aspect ratio increases.
For all but the highest aspect ratio datasets there exists a $\gamma$ such that for any
other $\gamma' > \gamma$ tested, a subspace embedding was returned for $\gamma$
and $\gamma'$.
However, on such high aspect ratio data (Slice ($d/n = 7 \times 10^{-3}$),
 Complex ($d/n=1.2 \times 10^{-1}$),and w1a ($d/n = 1.2 \times 10^{-1}$)) neither
 method performs particularly well with respect to returning a subspace embedding.
We further remark that on tall-and-skinny datasets CountSketch seems to return an
embedding at the same sampling factor as the SRHT with distortion that differs
by at most a small factor of $10^{-2}$ in the worst case (California Housing,
w3a), but more often on the order of $10^{-3}$.
One can see the progression of the rank retention through analysing the w$n$a
 datasets; neither method returns a full rank embedding for w$1$a at or above
 a particular sampling factor.
 Although both methods return a full rank embedding for w$3$a at $\gamma=5$,
 the w$4$a data requires $\gamma = 5$ for the CountSketch and $\gamma = 1$
 for the SRHT: this is the largest disparity between the two methods and occurs
 at $d/n = 4.1 \times 10^{-2}$.
 Despite the linear dependence on $\delta$ failure probability for a CountSketch
 subspace embedding compared to the $\log(1/\delta)$ of the other methods,
 the largest aspect ratio for which the CountSketch behaves almost identically
 to the SRHT is $d/n = 3 \times 10^{-2}$: less than this and we see only small
 differences in the distortion of the embedding.
 As a result, given the added benefit of a huge time saving, in the regime when
 $d/n < 3 \times 10^{-2}$, the CountSketch seems favourable for a subspace
 embedding becasuse it is fast and rank is retained at low sampling factors.
 If one were \textit{solely} interested in a subspace embedding but had
 slightly `fatter' data, along with less time constraints, then the SRHT is
 favourable.
 In spite of this, in the iterative framework a rank-deficient `embedding' might
 suffice providing that any lost directions in one iteration are picked up at
 later iterations.
 From this perspective, it appears that the CountSketch is favourable irrespective
 of the dimensionality because it is much faster and has comparable distortion
 to the SRHT even when rank is lost.

 We conclude the section by referring to the questions asked originally of the
 embedding behaviour.
 First we observed that the CountSketch embedding scaled with the density of the
 data but this growth was mild enough that for all aspect ratios tested it was
 a factor of 10 faster than the SRHT (which also scaled consistently with the
 theory).
 Secondly, for a variety of real datasets we have given a rough guide of how
 large the sampling factor $\gamma$ must be to ensure a subspace embedding is
 returned and shown examples when a `small' sampling factor is not sufficient.
 Finally, on the data we have tested, we have shown that there are examples
 when the CountSketch requires a larger embedding dimension to obtain a subspace
 embedding compared to the SRHT, however, this is only a small constant factor
 larger for the CountSketch despite the theory suggesting a factor $d$ larger.
 In addition, we have also shown empirically that there is a threshold below
 which the performance of the CountSketch is almost identical (in that rank is
 preserved and distortion is only a small constant factor different) to the SRHT
 but is vastly cheaper to compute.



\begin{figure}
        \centering
        \begin{subfigure}[b]{0.475\textwidth}
            \centering
            \includegraphics[width=\textwidth]{subspace_embedding_results/subspace_embedding_dimension_gaussian_2048_256.pdf}
            \caption{$d=256$}
            \label{fig: subspace-125e3-aspectratio}
        \end{subfigure}
        \hfill
        \begin{subfigure}[b]{0.475\textwidth}
            \centering
            \includegraphics[width=\textwidth]{subspace_embedding_results/subspace_embedding_dimension_gaussian_2048_512.pdf}
            \caption{$d=512$}
            \label{fig: subspace-375e1-aspectratio}
        \end{subfigure}
        \vskip\baselineskip
        \begin{subfigure}[b]{0.475\textwidth}
            \centering
            \includegraphics[width=\textwidth]{subspace_embedding_results/subspace_embedding_dimension_gaussian_2048_768.pdf}
            \caption{$d=768$}
            \label{fig: subspace-25e1-aspectratio}
        \end{subfigure}
        \quad
        \begin{subfigure}[b]{0.475\textwidth}
            \centering
            \includegraphics[width=\textwidth]{subspace_embedding_results/subspace_embedding_dimension_gaussian_2048_1024.pdf}
            \caption{$d=1024$}
            \label{fig: subspace-5e1-aspectratio}
        \end{subfigure}
        \caption{Distortion compared to sampling factor for various aspect ratios.
        Plotted are the mean results over 10 trials and a cross denotes when at least
        one trial has failed.
        The larger crosses indicate a higher number of failures with the largest
        crosses in Figure \ref{fig: subspace-5e1-aspectratio} for which CountSketch
        \textit{never} returned a full rank subspace embedding.}
        \label{fig: distortion-sampling-factor}
    \end{figure}



\section{Input Sparsity IHS} \label{sec: countsketch-ihs}
%!TEX root = sketching-convex-ols.tex


Although the IHS approach was studied empirically in \cite{pilanci2016iterative}
there was not a comparison of different sketching methods within the iterative
procedure.
This is of utmost importance to the practitioner because if one were to use a
Gausian sketch at every iteration then a prohibitive $O(nmd)$ time cost
would be inflicted per iteration.
Similarly, if one posesses sparse data then it was not known how a projection
which exploits sparsity behaves in comparison to the other sketching methods.
As such, we seek to understand the performance of the sketching methods
and compare the relative merits and tradeoffs.
Although the CountSketch performs favourably compared to the SRHT from a time
perspective, it does appear to have slightly more distortion variation in the
subspace embedding condition compared to both other methods.
Accordingly, one needs to understand whether this distortion is prohibitive in
the IHS scheme when using the CountSketch compared to other methods.
To aid our understanding of each of the sketching methods within the IHS we
first
reproduce some of the previous results from \cite{pilanci2016iterative} in order
to understand how each of the sketches behave.
This behaviour in this set of experiments is roughly similar to that found in
\cite{pilanci2016iterative}, although on occasion we need slightly different
parameter settings - this remains consistent with the theory, however.

The experiments in this section use synthetic data $A \in \R^{n \times d}$
whose entries are drawn from standard normal distribution.
A ground truth vector $x^*$ is chosen at random and then standard Gaussian
noise is added $\omega_i$ for $i=1, \ldots, n$.
The given target vector is then $b = Ax^* + \omega$




\textbf{Error compared to data size.}
The first experiment seeks to understand how the error responds when the size of
the data is increased.
The dimensionality of the dataset $A$ is fixed at $d=10$ and then more samples
are added with $n \in \{100 \times 2^i \text{ for } i \in \{3,4,\ldots,14\} \}$.
The number of iterations is fixed at $N=1+\log n$ and a sketch size for the IHS
is fixed at $m=5d$.
The experiments are repeated (insert number) times and the mean has been taken.
To maintain a comparison across a the same number of random projections, the
sketch-and-solve model is instantiated with sketch dimension set at $m'=Nmd$.
As shown in Figure \ref{fig: error-vs-row-dim}, the sketch-and-solve approach
is a suboptimal estimator of the true solution and this is further reflected in
the prediction error.
Despite only a modest increase in the number of iterations as $n$ grows larger,
the IHS estimates are approaching the optimal estimator as expected.
All three methods descend to the error at a rate consistent with the theory
(Proposition \ref{prop: ihs-error-bound}) and are each comparable with the
optimal estimator.
There appears to be little difference between the choice of random projection in
this setup and the faster CountSketch performs comparably to the Gaussian
 and
SRHT sketches which provides the first evidence that a cheaper to compute
projection may well prove fruitful.

\begin{figure}
  \centering
  \includegraphics[scale=0.75,keepaspectratio]{verify_ihs_error_num_rows}
  \caption{Solution Error vs Row Dimension and Prediction Error vs Row Dimension}
  \label{fig: error-vs-row-dim}
\end{figure}


\textbf{Solution Error vs Number of Iterations}
For this experiment we sampled data as above with $(n,d) = (6000,200)$
and tested sketch sizes $m = \gamma d$ for $\gamma = 4,6,8$.
The IHS method was run for $T=5,10,15, \ldots, 40$ iterations and for each
different sketch the error to the optimal estimator was measured (averaged
over ... repeats).
The results are displayed in Figure \ref{fig: ihs-sketch-errors}.
All three methods converge geometrically towards the optimal solution with
the rate increasing when larger $\gamma$ is used - this is illustrated in
Figure \ref{fig: error-to-lsq}.
The CountSketch follows a similar path to both other sketching methods
with slightly more (but not a prohibitive amount of) variation.
All three methods are relatively consistent in their performance however and
from this perspective it does not appear that a particular sketch is preferable
over any other.
Simiarly, in Figure \ref{fig: error-to-truth} we see how quickly the sketches
allow the IHS to converge to the optimal error between an estimator and the
ground truth (nb. for this experiment it is roughly $\log \sqrt{200/6000}
\approx -1.7$).
Again, as expected, a larger $\gamma$ enables faster convergence but from
this perspective the SRHT appears most stable with the least variation across
the iterations: both other methods seem to oscillate roughly a small constant
factor either side of the SRHT error.


\begin{figure}
    \centering
    \begin{subfigure}{0.49\textwidth}
    \centering
        \includegraphics[scale=0.5,keepaspectratio]{verify_ihs_error_to_lsq}
        \caption{Error to optimal estimator}
        \label{fig: error-to-lsq}
    \end{subfigure}%
    \begin{subfigure}{0.49\textwidth}
    \centering
        \includegraphics[scale=0.5,keepaspectratio]{verify_ihs_error_to_truth}
        \caption{Error to truth}
        \label{fig: error-to-truth}
    \end{subfigure}
    \caption{Solution error (compared to optimal estimator) and Error to
    ground truth vs number of
    iterations in Figures \ref{fig: error-to-lsq} and \ref{fig:
    error-to-truth}, respectively.
    The constant to the right of the sketch method in the legend
    is the sampling factor $\gamma$ of the sketch i.e. projection dimension
    $m = \gamma d$.}
    \label{fig: ihs-sketch-errors}
\end{figure}




\textbf{Error compared to dimensionality}
Here we generated data with $d$ varying across $\{ 16,32,64,128,256 \}$ and
taking $n = 250d$.
Again, $T = 1 + \log n$ iterations were used in the IHS with a sampling factor
of $\gamma=10$ (and $m = \gamma d$); hence the corresponding sketch-and-solve
sketch dimension is $T \gamma d$.
The optimal least squares estimator has error which is roughly

\begin{figure}
  \centering
  \includegraphics[scale=0.75,keepaspectratio]{verify_ihs_error_dimension}
  \caption{Solution error compared to optimal estimator}
  \label{fig: error-vary-columns}
\end{figure}

\subsection{Time}
Vary sparsity.
\subsection{Space}

\subsection{Distortion}







\section{Case Study: LASSO} \label{sec: ihs-lasso}
%!TEX root = sketching-convex-ols.tex

\textbf{Synthetic Data.}
\textit{Experimental Setup.}
For the purposes of this section we assume that data and targets are chosen
from the Gaussian design with unit covariance as described in Section
\ref{sec: why-ihs}.
We fix a positive $\lambda \in \R$ and use this to define an instance of LASSO
regression $x_{OPT} = \argmin_{x \in \R^d} 0.5 \|Ax - b\|_2^2 + \lambda \|x\|_1$.
Each iteration reduces the problem to a smaller instance of the problem which
is solved by quadratic programming using a method similar to \cite{gaines2018algorithms}.
We carry out two experiments: the first compares the (various metrics of) error
as the number of iterations is increased and the second measures the error
when the IHS model is run for a fixed amount of time.
Section
\ref{sec: subspace-embedding-results} showed that the CountSketch was roughly
an order of magnitude faster to compute but the compromise to be made was that
the embedding was less stable at retaining rank than the SRHT.
In this section we question (1) do we lose out by using an embedding which
preserves $\|Ax\|_2^2$ with slightly less fidelity?
(2) Upon understanding (1), how is this reflected in the run time of the
IHS model, in particular, how does the size of the sketch affect running time
and convergence of the IHS model?
(3) How do the methods scale as the data size is increased?






\noindent
\textbf{Error vs Number of Iterations.}
Results are plotted for the approximation $\hat{x}$ of an optimal solution $x_{OPT}$
to the lasso regression problem in Figure \ref{fig: ihs-lasso-error-2-opt}
The number of iterations has been varied from 4 to 16 and the mean solution error measured
over 5 independent runs of the IHS algorithm.
Each of the Gaussian, SRHT, and CountSketch was tested using sketch dimensions
of $m=4d, 6d, 8d$.
All three settings of $m$ exhibit similar behaviour across the three sketch
methods and the errors for each appear to be within a constant factor of
one another.
The largest difference in error is when $m=4d$ with the SRHT giving the best
approximation, but for each of the other settings the error converges to the
same point after 16 iterations, albeit with sharper decay with a larger sketch.
Despite the small changes in the solution error continuing up to 16 iterations,
Figure \ref{fig: ihs-lasso-error-2-truth} suggests that continuing beyond 12
iterations is only reducing the
error to the optimal approximation of the model (wrt the ground truth weights)
by a negligible amount.

\begin{figure}
        \centering
        \begin{subfigure}[b]{0.475\textwidth}
            \centering
            \includegraphics[width=\textwidth]{time_error_ihs_results/verify_ihs_error_to_opt}
            \caption{$n=50000$}
            \label{fig: ihs-lasso-error-2-opt}
        \end{subfigure}
        \hfill
        \begin{subfigure}[b]{0.475\textwidth}
            \centering
            \includegraphics[width=\textwidth]{time_error_ihs_results/verify_ihs_error_to_truth}
            \caption{$n=100000$}
            \label{fig: ihs-lasso-error-2-truth}
        \end{subfigure}
        \caption{nb finalise Fig 2b and remove on of the legends.}
        \label{fig: ihs-lasso-opt-truth}
\end{figure}



\noindent
\textbf{Wall Clock Time Complexity}
\textit{Experimental Setup.}
Only considering the number of iterations required to converge does not
emphasise the benefit of using the CountSketch embedding.
To elicit this computational boon we run the IHS model for a fixed period of
time ranging from 0.0005 seconds to 0.2 seconds and measure the accuracy of
the solution approximation.
Results are given in Figure \ref{fig: ihs-lasso-time-comparison} with
$n=50,000$ and $d=10$.
We tested 10 independent runs of the
algorithm each with a random permutation of the dataset and plotted the means.
In Figure \ref{table: real-data-subspace-embedding} we observed that using a
larger embedding dimension $m$ generally required more time but with the benefit
of obtaining a more accurate approximation.
Additionally, in Figure \ref{fig: summary-time-50000}, we also see the CountSketch
return an embedding an order of magnitude faster.
Here, in Figure \ref{fig: ihs-lasso-error-time}, we begin to see the benefits of
using the CountSketch come to fruition as we can compute an embedding so quickly
that
descent towards the optimum occurs before even one iteration of using the
SRHT can be carried out!
Although each iteration is `lower quality' when using the CountSketch (as seen
in Figure \ref{fig: ihs-lasso-opt-truth}) because they are so \textit{cheap}
in comparison the scheme can quickly approximate the solution.
Additionally, we also observe an interesting tradeoff that was not apparent when
considering only the number of iterations for convergence or the time cost of
computing an embedding: Figure \ref{fig: ihs-lasso-error-time} explicitly shows
a tradeoff between the selected embedding dimension and the time to converge.
Despite the embedding at $m=5d$ being cheaper to find, an embedding which is
slightly slower to find but is more accurate results in a faster run time overall.
Again, this is made apparent in Figure \ref{fig: ihs-lasso-iters-time} where
it is shown that the number of iterations using the CountSketch embedding
increases \textit{linearly} with the allotted time and there is negligible
difference between the number of iterations used at either $m=5d$ or $m=10d$;
the number of iterations required for the SRHT is also linear, albeit with a
\textit{significantly} reduced constant.



\begin{figure}
    \centering
    \begin{subfigure}[b]{0.475\textwidth}
        \centering
        \includegraphics[width=\textwidth]{time_error_ihs_results/error_time_50000_10}
        \caption{$n=50000$}
        \label{fig: ihs-lasso-error-time}
    \end{subfigure}
    \hfill
    \begin{subfigure}[b]{0.475\textwidth}
        \centering
        \includegraphics[width=\textwidth]{time_error_ihs_results/num_iters_time_50000_10}
        \caption{$n=100000$}
        \label{fig: ihs-lasso-iters-time}
    \end{subfigure}
    \caption{Error and Number of iterations compared to time.
              nb. change time setup to get smoother curves and test
              on different dimensionality datsets as well.
              Include Gaussian for comparison.
              Second figure is not log time.}
    \label{fig: ihs-lasso-time-comparison}
\end{figure}


\textbf{TBC.}

In this experiment we compare how the sparse projection methods compare with
other sketches in recovering the solution to a popular constrained regression
problem, the LASSO.
We test the IHS method on both synthetic and real datasets, showing that an approximate
estimator can be output in time faster than the commonly used sklearn
python machine learning library which has comparable performance.
The synthetic experiments also show parameter ranges for which the IHS method
is favourable in comparison to the exact method as well as highlighting, again,
the suboptimality of the sketch-and-solve model.

We generated a random regression instance as follows: the density of the data
was fixed at 10\%, ground truth weights $x^*$
and a design matrix $A \in \R^{n \times d}$ were generated each from a standard
normal distribution.
Uniformly at random, roughly 20\% of the entries in $w^*$ were chosen uniformly
at random to be set to zero.
Finally, we added Gaussian noise with unit variance to create the response values
$b = Ax + \omega$.
This dataset remained fixed and 5 trials of the sketching models were instantiated
with either the CountSketch or SRHT sketching methods.
We used $\log_{10}(n)$ iterations with a projection dimension $m = 2d$
To compare the performance of our implementations we solve the above model using
the \texttt{sklearn.linear\_model.Lasso} with a regularization parameter of 1 to
generate $x_{OPT}$.
Although we cannot exploit the penalised form of LASSO in our sketching models,
we took $R = \|x\|_1$ and used this as the $\ell_1$ norm constraint for each
of the QPs which the models are required to solve.
We report the means of solution error, cost function error, and time taken to
compute the optimisation.

\noindent\textbf{Time Complexity compared to dimensionality.}
The LASSO method scales as $O(nd^2)$ using the sklearn implementation of the ...
algorithm.
In comparison, the IHS method requires the time taken to compute $N$ sketches at
a cost of $T_{\text{sketch}}$ each, and $T_{\text{opt}}$ for every iteration.
A clear drawback of the IHS method is that for every iteration one needs to solve
a quadratic program in $d$ parameters so in the worst case costs $T_{\text{opt}}
= O(d^3)$ per iteration update.
As such, we need to understand how prohibitive this scaling is, and if so,
where exactly the IHS model is favourable compared to the exact method.
This behaviour is explored in Figure \ref{fig: lasso-synthetic} where each
sub-figure is the same experiment but with different $n$.
The experimental setup is to fix an $n \in \{ 50000, 100000, 200000, 400000 \}$
and then generate a dataset under the gaussian design with unit variance and
various $d$ from $10$ to roughly $300$.
Sketch sizes of $m = 2d$ were used throughout.
The density of the data was fixed at $\rho = 0.05$, the number of iterations was
fixed exactly at $\lceil \log_{10}(n) \rceil$ and the experiments were repeated 5 times (the
means have been plotted).
The combined time of sketch time and solve times have been plotted and
as expected, in each of the subfigures, there is a point beyond which the IHS
is slower than the exact method; shown by the dashed lines, this is due to the
cubic scaling of the quadratic programming rather than the sketching time which
increases only mildly as $d$ is increased.

\begin{figure}
        \centering
        \begin{subfigure}[b]{0.475\textwidth}
            \centering
            \includegraphics[width=\textwidth]{lasso_time_vs_cols_detailed50000}
            \caption{$n=50000$}
            \label{fig: lasso50000}
        \end{subfigure}
        \hfill
        \begin{subfigure}[b]{0.475\textwidth}
            \centering
            \includegraphics[width=\textwidth]{lasso_time_vs_cols_detailed100000}
            \caption{$n=100000$}
            \label{fig: lasso100000}
        \end{subfigure}
        \vskip\baselineskip
        \begin{subfigure}[b]{0.475\textwidth}
            \centering
            \includegraphics[width=\textwidth]{lasso_time_vs_cols_detailed200000}
            \caption{$n=200000$}
            \label{fig: lasso-200000}
        \end{subfigure}
        \quad
        \begin{subfigure}[b]{0.475\textwidth}
            \centering
            \includegraphics[width=\textwidth]{lasso_time_vs_cols_detailed400000}
            \caption{$n=400000$}
            \label{fig: lasso-400000}
        \end{subfigure}
        \caption{Time (log scale) of the IHS methods vs. Number of columns in
        the dataset.}
        \label{fig: lasso-synthetic}
    \end{figure}


\textbf{Real Datasets.}

\textit{Experimental setup.}
We carry out two experiments on the real datasets.
The first is to see how the number of iterations affects both the solution error,
the MSE, and the prediction error in both the IHS and sketch-and-solve models.
The second experiment compares the time cost of running the approximation
algorithms compared to that of sklearn for a fixed number of iterations with
a comparison of the error metrics.
In each experiment we separate the data into a train and test set of size 70\%
and 30\%, respectively.


\section{Case Study: SVM}

\subsection{Obtaining Machine Precision}

\subsection{Learning Rate}

\subsection{IHS as a fast approximate solver}


\appendix

\section{Structural Properties of CountSketch} \label{sec: countsketch-proofs}
\begin{Lemma} \label{lem: zero-mean}
  Entries of $S$ are zero-mean random variables.
\end{Lemma}

\begin{proof}
  For a fixed column $j$, the Count Sketch construction gives:

  \begin{equation}
    S_{ij}=
    \begin{cases}
      1, & \text{with probability}\ \frac{1}{2m} \\
      -1, & \text{with probability}\ \frac{1}{2m} \\
      0, & \text{with probability}\ \frac{m-1}{m} \\
    \end{cases}
  \end{equation}
  Then $\E S_{ij} = \frac{1}{2m}(1) + \frac{1}{2m}(-1) + \frac{m-1}{m}(0) = 0$.
\end{proof}

\begin{Lemma} \label{lem: orthogonal_rows}
  Let $S$ be a CountSketch matrix.
  Let $N_i$ denote the number of nonzeros in row $S_i$.
  Then $SS^T$ is a diagonal matrix with $(SS^T)_{ii} = N_i$ and in particular,
  distinct rows of $S$ are orthogonal.
\end{Lemma}

\begin{proof}
  The matrix $SS^T$ has entries whose values are the inner products between any
  two rows of $S$ so consider the inner product $\langle S_i, S_j \rangle$.
  For distinct rows $S_i$ and $S_j$ if the inner product is non-zero then there
  must be some term in the inner product, $S_{ik} S_{jk}$, which is non-zero.
  However, this would imply that both $S_{ik}$ and $S_{jk}$ are non-zero which
  is not possible since only one row is chosen to be nonzero for every column
  $k$.
  The diagonal entries are simply $\| S_i \|_2^2 = \sum_{j=1}^n S_{ij}^2$ and
  $S_{ij}^2 = 1$ for every nonzero entry in row $S_i$ so summing over all of
  the columns $j$ gives $N_i$ as required.
\end{proof}

\begin{Lemma}
  $\E(SS^T) = \frac{n}{m} I_m$
\end{Lemma}

\begin{proof}
  Observe that $(S S^T)_{ij} = \langle S_i, S_j \rangle$ which is the inner
  product between rows of $S$.
  Then:

  \begin{align}
    \E (S_i \cdot S_j) &= \E \left( \sum_{k=1}^n S_{ik} S_{jk} \right) \\
                       &= \sum_{k=1}^n \E (S_{ik} S_{jk}).
  \end{align}
  For a fixed column $k$, if row $i \ne j$ then by Lemma \ref{lem:
  orthogonal_rows} we know that $\langle S_i, S_j \rangle = 0$ and
  hence $\E (S_i \cdot S_j) = 0.$
  Otherwise, $i=j$ and this is exactly the row norm of $S_i$ which in expectation
  is $n/m$.
  This is because $S_{ik}^2 = 1$ with probability $1/m$ and 0 otherwise
  so that $\E S_{ik}^2 = 1/m$.
  Hence, by linearity we obtain the following which proves the claim.

  \begin{equation}
    \E (S_i \cdot S_i) = \sum_{k=1}^n \E S_{ik}^2 =
     \frac{n}{m}.
  \end{equation}
\end{proof}

\begin{Lemma} \label{lem: covariance_matrix}
  The covariance is identity: $\E(S^TS) = I_{n \times n}$
\end{Lemma}

\begin{proof}
  Observe that $\E(S^T S)_{ij} = \langle S^T_i, S^j \rangle = \langle S^i, S^j
  \rangle$ so we can consider only dot products betwen columns of $S$.
  For a particular entry in the inner product the terms are $S_{ki}S_{kj}$ which
  is 1 only when both $S_{ki}$ and $S_{kj}$ are the same sign, -1 when both
  terms are opposite sign, and 0 if either of the two is 0.

  If $i=j$ then the inner product is exactly the norm of a column of $S$
  which, by construction, is exactly 1.
  If $i \ne j$ then we can use linearity of expecation and independence of the
  random variables to consider only $\E(S_{ik} S_{jk}) = \E(S_{ik}) \E(S_{jk})
  = 0$ by Lemma \ref{lem: zero-mean}.
  This shows that

  \begin{equation*}
    \E (S^T S)_{ij} =
    \begin{cases}
      1, & \text{if}\ i = j \\
      0, & \text{if}\ i \ne j, \\
    \end{cases}
  \end{equation*}
  which is exactly the $n \times n$ identity matrix.

  % Also, if $i=j$
  %
  % Now assume that $i \ne j$ so that we have the following:
  %
  % \begin{equation}
  %   S_{ki} S_{kj} =
  %   \begin{cases}
  %     1, & \text{if}\ S_{ki} = S_{kj} \\
  %     -1, & \text{if}\ S_{ki} \ne S_{kj} \text{but}\ k = h(i) = h(j) \\
  %     0, & \text{otherwise} \\
  %   \end{cases}
  % \end{equation}
  % So $S_{ki} S_{kj} = 1$ with probability $1/2m$, $S_{ki} S_{kj} = -1$ with
  % probability $1/2m$ and is 0 with probability $(m-1)/m$.
  % Then taking expectation
\end{proof}





\begin{mydef}
  A zero-mean random vector $s \in \R^n$ is 1-sub-Gaussian if for any $u \in \R^n$
  we have for all $\eps > 0$

  \begin{equation}
    \prob [ \langle s, u \rangle \ge \eps \| u \|_2 ] \le \exp(-\eps^2/2).
  \end{equation}
\end{mydef}

\begin{rem}
  If, instead, $\frac{1}{\sqrt{m}}$-sub-Gaussian distributions are necessary
  then one can sample from $S'$ a $1$-sub-gaussian distribution and perform the
  normalisation $S = S'/\sqrt{m}$.
\end{rem}

\begin{Lemma}[Rows of CountSketch are sub-Gaussian]
  Let $S$ be an $m \times n$ random matrix sampled according to the CountSketch
  construction.
  Then any row $S_i$ of $S$ is 1-sub-Gaussian.
\end{Lemma}

\begin{proof}
  Fix a row $S_i$ of $S$ and let $X = \langle S_i, u \rangle$.
  If either $S_i$ or $u$ is a zero vector then the
  inequality in the sub-Gaussian definition will always be true so assume that
  this is not the case.
  We need the following version of Bernstein's Inequality:
  when $X$ is a sum of $X_1,\ldots,X_n$ and $|X_j| \le M$ for all $j$ then for
  any $t > 0$

  \begin{equation} \label{eq: Bernstein}
    \prob (X > t) \le \exp \left( - \frac{t^2 / 2}{\sum_j \E X_j^2 + Mt/3
    } \right).
  \end{equation}
  Now, $X = \sum_{j=1}^n S_{ij} u_j$ which is a sum of zero-mean random variables.
  Also, $|X_j| = |S_{ij} u_j| \le \| u \|_2$ for every $j$ and $\E X_j^2 =
  u_j^2/m$.
  Taking $t = \eps \| u \|_2$ in Equation (\ref{eq: Bernstein}) and cancelling
  $\| u \|_2^2$ terms gives

  \begin{equation} \label{eq: countsketch_bernstein}
    \prob (X > \eps \| u \|_2) \le \exp \left( - \frac{\eps^2 /
    2}{1/m + \eps /3 } \right)
    \le \exp(-\eps^2/2).
  \end{equation}
  The final inequality holds whenever $1/m + \eps / 3 \le 1$ and indeed this is
  true whenever $\eps \le 3 - 3/m$.
  However, since $\eps \in (0,1)$ the inequality holds
  for all choices of $\eps$ provided $m>1$.
  % We need the following version of Chernoff bound when $X$ is a sum of $X_1,
  % \ldots,
  % X_n$, for any $t > 0$ (can alter to minimal and product of expectations by
  % independence if need be):
  %
  % \begin{equation}
  %   \prob (X > a) \le \exp(-ta) \E \left( \prod_{i=1}^n e^{tX_i} \right)
  % \end{equation}
  %
  % Let $X_j = S_{ij}u_j$ which are independent by the independence of $S_{ij}$
  % and observe that $\E e^{t X_i} = \E e^{t S_{ij} u_j}$.
  % The $S_{ij} u_j$ terms are $\pm u_j$ with probability $1/2m$ each, or 0 with
  % probability $\frac{m-1}{m}$ so that:
  %
  % \begin{equation}
  %   \E e^{t S_{ij} u_j} = \frac{e^{tu_j} + e^{-tu_j}}{2m} + \frac{m-1}{m}e^{t 0}.
  % \end{equation}
  %
  % Now, $\frac{e^{tu_j} + e^{-tu_j}}{2m} = \frac{1}{m} \cosh(tu_j)$.
  % Also, $\cosh(t u_j)$ is at least one for all real inputs so we also have
  % $\frac{m-1}{m} = \frac{m-1}{m}(1) \le \frac{m-1}{m} \cosh(tu_j)$.
  % Summing both of these terms gives $\E e^{t X_j} \le \cosh(tu_j)$ so that
  % $\prod_{j=1}^n \E e^{t X_j} \le \prod_{j=1}^n \cosh(tu_j)$.
  % In addition we also have (standard bound?) that $\cosh(tu_j) \le \exp(t^2
  % u_j^2 / 2)$ which gives $\prod_{j=1}^n \E e^{t X_j} \le \prod_{j=1}^n \exp(t^2
  % u_j^2 / 2) = \exp(t^2 \|u\|_2^2/2)$.
  %
  % Hence, by the Chernoff bound we obtain $\prob (X \ge a) \le \exp(t^2 \|u
  % \|_2^2 / 2 -ta)$.
  % Taking $a = \eps \| u \|_2$ we see that
  %
  % \begin{equation} \label{eq: chernoff2subgauss}
  %   \prob ( X \ge \eps \|u \|_2 ) \le \exp \left(\frac{t^2 \| u \|_2^2 }{2}
  %   - t \eps \| u \|_2 \right).
  % \end{equation}
  %
  % However, for in order to prove the sub-Gaussian property we need Equation
  % \ref{eq: chernoff2subgauss} to be bounded above by $\exp(-\eps^2/2)$.
  % Since $\exp(-y)$ is decreasing in $y$, this amounts to ensuring that
  %
  % \begin{equation*}
  %   \frac{t^2 \|u \|_2^2}{2} - t \eps \|u \|_2 \ge \frac{\eps^2}{2}
  % \end{equation*}
  % which is true provided that $t \ge \frac{(1 + \sqrt{2}) \eps}{\|
  % u \|_2}$ (positive solution)and hence for any $\eps > 0$ there is a $t > 0$ for
  % which the sub-Gaussian property is true.

\end{proof}

\subsection{Spectral Bound}
Here we need to show that for a constant $\eta$ which is independent of $m$
and $n$ we have:

\begin{equation} \label{eq: spectral_bound}
  \| \E \left( S^T (S S^T)^{-1} S \right) \|_2 \le \eta \frac{m}{n} I_n.
\end{equation}
Before proving this statement for the CountSketch transform we give the following
lemma which is necessary to conclude the analysis.

\begin{Lemma} \label{lem: exp_recip_bernoulli}
  Let $X = \sum_{i=1}^n X_i$ where each $X_i$ is iid Bernoulli rv with
  parameter $p$.
  Then
  \begin{equation}
    \E \left( \frac{1}{1+X} \right) = \frac{1 - (1-p)^{n+1}}{(n+1)p}
  \end{equation}
\end{Lemma}

\begin{proof}
  Observe that $X$ is a sum of Bernoulli rvs so has a Binomial probability
  mass function, hence:

  \begin{align}
    \E \left(  \frac{1}{1+X} \right) &= \sum_{k=0}^n \frac{1}{1+k}
                {n \choose k} p^k (1-p)^{n-k} \\
                &= \frac{1}{p(n+1)} \sum_{k=0}^n {n+1 \choose k+1} p^{k+1}
                (1-p)^{n-k} \\
                &= \frac{1 - (1-p)^{n+1}}{p(n+1)}.
  \end{align}
The last equality holds by the Binomial Theorem.


% PROOF OUTLINE FOR THE BINOMIAL THEOREM DEDUCTION.
% Write $1 = (p + (1-p))^{n+1}$ and then expand by the Binomial Theorem.
%
% \begin{align}
%   1 &= \sum_{j=0}^{n+1} {{n+1 \choose j}} p^j (1-p)^{n+1-j} \\
%     &= \sum_{j=1}^{n+1} {{n+1 \choose j}} p^j (1-p)^{n+1-j} + (1-p)^{n+1} \\
%     &= \sum_{k=0}^{n} {{n+1 \choose {k+1}}} p^{k+1} (1-p)^{n-k} + (1-p)^{n+1}.
% \end{align}
% The final equality is deduced by setting $j = k+1$ and this gives
% $1 - (1-p)^{n+1} = \sum_{k=0}^{n} {{n+1 \choose {k+1}}} p^{k+1} (1-p)^{n-k}$ as
% required.
% Alternatively, if $P_X(t)$ is the probability generating function of $X$ then:
%
%   \begin{equation}
%     \E \left(  \frac{1}{X+a} \right) = \int_0^1 t^{a-1} P_X(t) dt
%   \end{equation}
% Can follow this through for another proof.
\end{proof}

\begin{thm} \label{thm: spectral-theorem}
  Let $S$ be an $m \times n$ CountSketch matrix.
  Assume that $S$ contains no zero rows.
  Then the spectral bound (\ref{eq: spectral_bound}) is met with $\eta = 1$.
\end{thm}

\begin{proof}
  We need to analyse the quantity $\E \left( S^T (S S^T)^{-1} S \right)$ and
  show this is a diagonal matrix whose entries are a constant multiple of $m/n$.
  First we deal with the term $SS^T$ which by Lemma \ref{lem: orthogonal_rows}
 is a diagonal matrix whose entries are $N_i$, the number of nonzeros in row
  $S_i$.
  % Observe that $SS^T$ is the matrix containing all inner products between rows
  % of $S$.
  % By Lemma \ref{lem: orthogonal_rows} we know that any cross terms are
  % identically zero so the only non-zero terms are those on the diagonal.
  % For a suitably chosen range of $m$ and $n$ (to be made precise at some
  % point) we may assume
  % that all rows are non-zero and hence the diagonal entries are the
  % norms of the rows of $S$, which are thus all greater than zero.
  Hence we may take $D = SS^T$ with entries $D_{ii} = N_i$.
  Since all the diagonal entries in $D$ are non-zero, $D^{-1}$ exists and is
  precisely $D^{-1}_{ii} = 1/ N_i$. %$\| S_i \|_2^2$.
  Now, we may write $S^T (SS^T)^{-1} S = S^T D^{-1} S $ and distribute the
  entries of $D^{-1}$ as $S^T D^{-1/2} D^{-1/2} S$ which is well-defined as
  $D^{-1}$ is both diagonal and positive.
  Set $\bar{S} = D^{-1/2} S$ which is simply $S$ but scaled by its row norms as:

  \begin{equation} \label{eq: S-bar_vals}
    \bar{S}_{ij} = \frac{S_{ij}}{N_i^{1/2}}.
  \end{equation}
  So consider the matrix $\bar{S}^T \bar{S}$ which is the collection
  of scaled inner products between the columns of $S$.
  In particular, for every column $\bar{S}^j$ the nonzero entry is located at
  $\bar{S}_{h(j),j}$ and takes value $S_{h(j),j}/N_{h(j)}^{1/2}$.

  % First, suppose that $i \ne j$ and deal with the cross terms.
  % Since every column of $S$ contains exactly one non-zero entry the product
  % $\langle \bar{S}^i, \bar{S}^j \rangle$ will often be zero.
  % In particular, the inner product is zero whenever the columns are
  % hashed to different rows with certainty and is non-zero when $h(i) = h(j)$.
  % In this case, $\langle \bar{S}^i, \bar{S}^j \rangle$ has exactly one non-zero
  % element in the same row location so the inner product is $\sigma(i) \sigma(j)$
  % normalised by $\| S_{h(i)} \|_2 \| S_{h(j)} \|_2  =\| S_{h(i)} \|_2^2$.
  % Both of these scenarios are zero in expecation; the first being always zero,
  % and the second by 2-wise independence of the sign function $\sigma(\cdot)$.

  Since $\bar{S}$ has the same sparsity structure as $S$
  % (i.e. the nonzeros of
  % $\bar{S}$
  % and $S$ are located in exactly the same locations)
  and its entries are just
  rescaled versions of those in $S$, we know from Lemma \ref{lem:
  covariance_matrix} that in expectation the only entries we need to consider
  are those lying on the diagonal.
  Then the inner product to consider is $\langle \bar{S}^i, \bar{S}^i \rangle$
  which  is exactly the norm of column $\bar{S}^i$.
  Equation (\ref{eq: S-bar_vals}) shows that $(\bar{S}^T \bar{S})_{ii} = 1/
  N_{h(i)}$ and
  % Ordinarily the inner product of two columns would be
  %  1 but is now rescaled by a $1/ N_{h(i)}^{1/2}$ term from both $\bar{S}^T$
  %  and $\bar{S}$ so now the diagonal entries of $\bar{S}^T \bar{S}$ are
  % $1/ N_{h(i)}$.
  Lemma \ref{lem: exp_recip_bernoulli} can be used to bound $1/ N_{h(i)}$ in
  expectation
  %Indeed, $N_{h(i)} = \| S_{h(i)} \|_2^2$
  as it is a sum of Bernoulli random variables.
  Indeed, $N_{h(i)} = \sum_{j=1}^n \mathbf{1}(S_{h(i),j}^2)$ where $\mathbf{1}
  (S_{h(i),i})$ is an indicator variable for the presence of a nonzero in entry
  $S_{h(i),j}$.
  No row of $S$ is identically zero so certainly $S_{h(i), i}^2 = 1$ and for
  any $j \ne i$ then $\mathbf{1}(S_{h(i),j}) = 1$ with probability $1/m$; for a
  given column rows are chosen uniformly at random in the CountSketch
  construction.
  This is exactly when $h(j) = h(i)$ so:

  \begin{equation} \label{eq: split_row}
    N_{h(i)} = \sum_{j=1}^n S_{h(i), j}^2 = 1 + \sum_{\substack{ j \ne
     i \\ h(j) = h(i)}} S_{h(i), j}^2.
  \end{equation}
  Hence, we may take $N_{h(i)} = 1+X$ as in the final equality of
  Equation (\ref{eq: split_row}) and
  invoke Lemma \ref{lem: exp_recip_bernoulli} with $p=1/m$ and $n-1$
  trials (over all columns $j \ne i$ of $S$ as column $i$ contributes the
  1 in the expression $1+X$) to see:

  \begin{align}
    \E \left( \frac{1}{1+X} \right) &= \frac{1 - (1-\frac{1}{m})^n}{n/m} \label{eq: expectation}\\
                                &\le \frac{1}{n/m} = \frac{m}{n} \label{eq: final-expectation}.
  \end{align}
Overall we obtain (with the $i \ne j$ case holding with equality):

  \begin{equation}
    \E \left( \bar{S}^T \bar{S} \right)_{ij} \le
    \begin{cases}
      \frac{m}{n}, & \text{if}\ i=j \\
      0, & \text{if}\ i \ne j.
    \end{cases}
  \end{equation}
  Finally, this implies that $\E (S^T (SS^T)^{-1}S) \preceq \frac{m}{n} I_n$ and
  hence the spectral norm is bounded above by $m/n$ so property
  (\ref{eq: spectral_bound}) holds with $\eta = 1$ as claimed.

\end{proof}

% \begin{Cor} \label{cor: unbiased-sketch}
%   If $\theta = n/m$ then $\E[ \theta (S^T (SS^T)^{-1}S)]\mathbf{1} = \mathbf{1}$.
% \end{Cor}
%
% \begin{proof}
%   We claim that choosing $\theta = n/m$ with certainty is sufficient.
%   The proof follows that of Theorem \ref{thm: spectral-theorem} and observing
%   that $\theta$ is a scalar so can be absorbed into the matrices by altering
%   Equation \ref{eq: S-bar_vals} so that instead $\bar{S}_{ij} = \theta^{1/2}S_
%   {ij}/N_i^{1/2}$.
%   Since $\theta = n/m$ is now deterministic, we may follow the proof up to
%   Equation \ref{eq: expectation} and pulling the constant out to get $\theta m/n
%   $ in Equation \ref{eq: final-expectation}.
%   Again, because $\theta$ is deterministic $\bar{S}^T \bar{S}$ is diagonal in
%   expectation so it suffices to check the diagonal terms which are $\theta m/n$
%   and hence we may take $\theta = n/m$ to ensure that the matrix product with
%   the all-ones vector gives the required result.
% \end{proof}
%
% \begin{rem}
%   Corollary \ref{cor: unbiased-sketch} shows that there exists a $\theta$ under
%   which the action of the sketch projection recovers the all-ones vector in
%   expectation.
%   Now, treating $\theta$ as a random variable which is identically $n/m$ with
%   certainty, Corollary \ref{cor: unbiased-sketch} demonstrates that the CountSketch
%   is an \textit{unbiased sketch} and can be used in any result of \cite{gower2018stochastic}
%   which uses the identity as a weight matrix.
% \end{rem}


\bibliography{references}
\bibliographystyle{plain}
\end{document}
